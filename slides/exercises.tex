\documentclass{article}
\usepackage[a4paper,margin=1in]{geometry}
\usepackage{fancyhdr}
\usepackage{tikz}
\usepackage{xcolor}
\usepackage[ddmmyyyy]{datetime}

\usepackage{multicol}
\usepackage{amsmath}
\usepackage{amssymb}
\usepackage{amsthm}
\usepackage{pdfpages}

\usepackage{titlesec}


%hyperref should be last apparently
\usepackage{hyperref}

% Starts a new paragraph without indentation
% and with an empty line between paragraphs
\newcommand*{\newpar}{\par\vspace{\baselineskip}\noindent}
\newcommand{\ttt}[1]{\texttt{#1}}
\newcommand{\tbf}[1]{\textbf{#1}}
\newcommand{\ul}[1]{\underline{#1}}
\newcommand{\seperator}{\vspace{-8pt}\hrule\vspace{8pt}}
\newcommand{\id}{\text{id}}

\newcommand{\bAda}{{\mathbb{A}da}}
\newcommand{\bAdaP}{\mathbb{A}daP}
\newcommand{\bC}{\mathbb{C}}
\newcommand{\bD}{\mathbb{D}}
\newcommand{\bHask}{{\mathbb{H}ask}}
\newcommand{\bM}{\mathbb{M}}
\newcommand{\bPre}{\mathbb{P}re}
\newcommand{\bProf}{{\mathbb{P}rof}}
\newcommand{\bS}{\mathbb{S}}
\newcommand{\bSet}{\mathbb{S}et}
\newcommand{\nat}{\text{Nat}}

\newcommand{\mY}{\mathcal{Y}}

\titleformat{\section}
  {\normalfont\large\bfseries}{\thesection.}{1em}{}

%\lhead{Emma Bach}
%\rhead{\today}

%\pagestyle{fancy}

\begin{document}
% \newtheorem{codename}{printedname}[countedwith]
\newtheorem{lemma}{Lemma}
\newtheorem{theorem}[lemma]{Satz}
\newtheorem{proposition}[lemma]{Proposition}
\newtheorem{anmerkung}[lemma]{Anmerkung}

\theoremstyle{definition}
\newtheorem{definition}[lemma]{Definition}
\newtheorem{example}[lemma]{Example}


\title{\Large\vspace{-1.5cm}\textbf{What you needa know about Yoneda - Exercises}}
\author{Emma Bach, Seminar on Functional Programming and Logic}
\date{}

\maketitle
\begin{definition}
	A \textit{category} $\bC$ consists of:
	\begin{itemize}
		\item a collection $|\bC|$ of \textit{objects};
		\item for all $A, B \in |\bC|$, a collection $\bC(A, B)$ of \textit{morphisms} from $A$ to $B$;
		\item for all $A \in |\bC|$, an \textit{identity morphism} $id_A \in \bC(A, A)$;
		\item for each pair of morphisms $g \in \bC(B,C)$, $f \in \bC(A,B)$, a morphism $g \circ f \in \bC(A,C)$, such that composition is associative.
	\end{itemize}
\end{definition}
\begin{definition}
	A \textit{functor} $F : \bC \to \bD$ is a structure-preserving map between two categories:
	\begin{itemize}
		\item $F$ maps an object $A \in |\bC|$ to an object $F(A) \in |\bD|$
		\item $F$ maps a morphism $f \in \bC(A,B)$ to a morphism $F(f) \in \bD(F(A),F(B))$
		\item $F(id_A) = id_{F(A)}$
		\item $F(g \circ f) = F(g) \circ F(f)$
	\end{itemize}
\end{definition}
\begin{definition}
	A \textit{natural transformation} is a structure-preserving map between functors:
	\begin{itemize}
		\item Let $F,G : \bC \to \bD$ be functors.
		\item A natural transformation $\phi$ is an \textit{indexed family of morphisms} - for every object $A \in |\bC|$,  $\phi_A$ is a morphism from $F(A)$ to $G(A)$.
		\item These morphisms satisfy the \textit{naturality condition}:
		\begin{align*}
			\forall f \in \bC(A,B) : \phi_B \circ F(f) = G(f) \circ \phi_A
		\end{align*}
	\end{itemize}
\end{definition}
\section*{Exercise 1}
\seperator
\begin{itemize}
	\item[a)] Let $\leq$ be a reflexive, transitive order (a \textit{preorder}) on a set $M$. Show that if we define objects by $|\bPre(M, \leq)| = M$ and morphisms by $\exists! f_{x \leq y} \in \bPre(x,y) \Leftrightarrow x \leq y$, then $\bPre(M, \leq)$ forms a category.
	\item[b)] Let $F : \bM \to \bM$ be an endofunctor on $\bM$. Show that $F$ defines a monotonic function $M \to M$, i.e. $\forall x,y : x \leq y \implies F(x) \leq F(y)$.
	\item[c)] Let $F,G : M \to M$ be monotonic functions. Let $\phi$ be a natural transformation $F \to G$. Show that $\forall x \in M: F(x) \leq G(x)$.
\end{itemize}
%
%
%
%
%
\clearpage
%
%
%
%
%
\begin{definition}
	The \textit{homfunctor} $\bC(A, -)$ is defined as follows:
	\begin{itemize}
		\item An object $B$ is mapped to the homset $\bC(A,B)$.
		\item A morphism $f : \bC(B,C)$ is mapped to the morphism $f \circ : \bC(A,B) \to \bC(A,C)$.
	\end{itemize}
\end{definition}
\begin{definition}
	\textbf{Yoneda Lemma:} There exists a natural isomorphism 
	\[\nat(\bC(A,-), F) \simeq F(A)\]
\end{definition}
\begin{definition}
	\textbf{Yoneda Embedding:} For every object $A \in |\bC|$, there exists a bijective functor \[\mY : \bC \to \bC(A,-)\]
\end{definition}
\begin{proposition}
	Any monoid $M = (M, *, e)$ can be represented as a category $\bM$ with a single object $*$, where monoid elements are morphisms, such that composing two morphisms $m$ and $n$ gives the morphism corresponding to the element $m * n$.
\end{proposition}
\section*{Exercise 2}
\seperator
Use the Yoneda embedding to show that every monoid $M$ is isomorphic to a monoid of functions $M \to M$. 
\end{document}
