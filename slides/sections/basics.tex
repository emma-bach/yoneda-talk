\begin{frame}{Motivation}
 \begin{itemize}
  \item A common sentiment in many cultures is the idea that people are defined by how they interact with their surroundings.
  \pause\item ``\textit{Tell me your company, and I will tell you what you are.}''\footnote{Quoted as a proverb in \textit{Don Quixote}}
  \pause\item The Yoneda Lemma is the result of applying this way of thinking to mathematical objects within the extremely general setting of \textit{category theory}.
  \pause\item As a result, a category $\bC$ is often best understood by instead studying functors from that category into $\mathbb{S}et$.
 \end{itemize}
\end{frame}

\begin{frame}[fragile]{Categories}
\begin{columns}
\column{0.3\textwidth}
% https://q.uiver.app/#q=WzAsNCxbMCwwLCJBIl0sWzIsMCwiQiJdLFsyLDIsIkMiXSxbNywyXSxbMCwxLCJmIl0sWzEsMiwiZyJdLFswLDIsImYgXFxjaXJjIGciLDJdLFswLDAsImlkXzEiXSxbMSwxLCJpZF8yIl0sWzIsMiwiaWRfMyIsMix7InJhZGl1cyI6LTN9XV0=
\[\begin{tikzcd}
	A && B \\
	\\
	&& C &&&&& {}
	\arrow["{id_1}", from=1-1, to=1-1, loop, in=55, out=125, distance=10mm]
	\arrow["f", from=1-1, to=1-3]
	\arrow["{f \circ g}"', from=1-1, to=3-3]
	\arrow["{id_2}", from=1-3, to=1-3, loop, in=55, out=125, distance=10mm]
	\arrow["g", from=1-3, to=3-3]
	\arrow["{id_3}"', from=3-3, to=3-3, loop, in=305, out=235, distance=10mm]
\end{tikzcd}\]
\column{0.7\textwidth}
\begin{itemize}
 \item


A \textit{category} $\bC$ consists of:
\begin{itemize}
 \pause\item a collection $|\bC|$ of \textit{objects};
 \pause\item for all $A, B \in |\bC|$, a collection $\bC(A, B)$ of \textit{morphisms} from $A$ to $B$;
 \pause\item for all $A \in |\bC|$, an \textit{identity morphism} $id_A \in \bC(A, A)$;
 \pause\item an associative \textit{composition morphism} $f \circ g \in \bC(A,C)$ for each pair of morphisms $f \in \bC(A,B)$, $g \in \bC(A,B)$.
\end{itemize}
\pause\item If $\bC(A, B)$ is a set, we call it the \textit{homset} from $A$ to $B$.
\pause\item For every category $\bC$, there exists an \textit{opposite category} $\bC^{op}$, in which all morphisms are reversed.
\end{itemize}
\end{columns}
\end{frame}
\begin{frame}[fragile]{Homfunctors}
\begin{columns}
\column{0.3\textwidth}
 % https://q.uiver.app/#q=WzAsMyxbMCwwLCJBIl0sWzIsMiwiQyJdLFsyLDAsIkIiXSxbMCwxLCJmIFxcY2lyYyBnIiwxXSxbMCwyLCJmIiwxXSxbMiwxLCJnIiwxXV0=
\[\begin{tikzcd}
	A && B \\
	\\
	&& C
	\arrow["f"{description}, from=1-1, to=1-3]
	\arrow["{f \circ g}"{description}, from=1-1, to=3-3]
	\arrow["g"{description}, from=1-3, to=3-3]
\end{tikzcd}\]
% https://q.uiver.app/#q=WzAsNCxbNCwwXSxbMCwwLCJcXHtpZF9BXFx9Il0sWzIsMCwiXFx7ZlxcfSJdLFsyLDIsIlxce2YgXFxjaXJjIGdcXH0iXSxbMSwyLCJmIFxcY2lyYyIsMV0sWzIsMywiZyBcXGNpcmMiLDFdLFsxLDMsImYgXFxjaXJjIGcgXFxjaXJjIiwxXV0=
\[\begin{tikzcd}
	{\{id_A\}} && {\{f\}} && {} \\
	\\
	&& {\{f \circ g\}}
	\arrow["{f \circ}"{description}, from=1-1, to=1-3]
	\arrow["{f \circ g \circ}"{description}, from=1-1, to=3-3]
	\arrow["{g \circ}"{description}, from=1-3, to=3-3]
\end{tikzcd}\]
\column{0.7\textwidth}
 \begin{itemize}
  \item For any category $\bC$, a homset $\bC(A,B)$ is a set of morphisms.
  \pause\item We define a functor $\bC(A, -) : \bC \to \mathbb{S}et$:
  \begin{itemize}
    \pause\item $\bC(A, -)$ maps an Object $B$ to the Homset $\bC(A,B)$
    \pause\item A morphism $f : \bC(B,C)$ is mapped to the morphism $f \circ : \bC(A,B) \to \bC(A,C)$
  \end{itemize}
  \pause\item Similarly, $\bC(-, B)$ is a functor $\bC^{op} \to \mathbb{S}et$:
  \begin{itemize}
    \pause\item $\bC(-, B)$ maps an Object $A$ to the Homset $\bC(A,B)$
    \pause\item A morphism $f : \bC(A,B)$ is mapped to the morphism $\circ f: \bC(B,C) \to \bC(A,C)$
  \end{itemize}
 \end{itemize}
 \end{columns}
\end{frame}
\begin{frame}{Natural Transformations}
    \begin{itemize}
     \item A structure-preserving map between functors.
     \begin{itemize}
      \item Let $F,G : \bC \to \bD$ be functors.
      \item A natural transformation $\phi$ is an indexed family of morphisms $\phi_A \in \bD(F(A), G(A))$ from $F(A)$ to $G(A)$
      \item These morphisms satisfy the following \textit{naturality condition}:
      \begin{align*}
       \forall f \in \bC(A,B) : \phi_B \circ F(f) = G(f) \circ \phi_A
      \end{align*}
     \end{itemize}
     \item Given two functors $F$ and $G$, we write the collection of all natural transformation between them as $\text{Nat}(F, G)$.
    \end{itemize}
\end{frame}
\begin{frame}{Naturality from Polymorphism}
    \begin{itemize}
      \item The naturality condition resembles an equality we saw a few weeks ago:
      \begin{align*}
       r_B \circ \texttt{map}(a) = \texttt{map}(a) \circ r_A
      \end{align*}
      \pause \item This is the free theorem we got for a parametrically polymorphic function \texttt{r :: [X] -> [X]} and an arbitrary function \texttt{a : A -> B}.
      \pause \item This free theorem is equivalent to the statement that $r$ is a natural transformation.
    \end{itemize}
\end{frame}
