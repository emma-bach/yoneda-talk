\begin{frame}{Motivation}
 \begin{itemize}
  \item A common sentiment in many cultures is the idea that things are defined by how they interact with their surroundings.
  \pause\item ``\textit{Tell me your company, and I will tell you what you are.}''\footnote{Quoted as a proverb in \textit{Don Quixote}}
  \pause\item The Yoneda Lemma is the result of applying this way of thinking to mathematical objects in category theory.
  \pause\item As a result, a category $\bC$ is often best understood by instead studying functors from that category into $\mathbb{S}et$.
 \end{itemize}
\end{frame}

\begin{frame}[fragile]{Categories}
\begin{columns}
\column{0.3\textwidth}
% https://q.uiver.app/#q=WzAsNCxbMCwwLCJBIl0sWzIsMCwiQiJdLFsyLDIsIkMiXSxbNywyXSxbMCwxLCJmIl0sWzEsMiwiZyJdLFswLDIsImYgXFxjaXJjIGciLDJdLFswLDAsImlkXzEiXSxbMSwxLCJpZF8yIl0sWzIsMiwiaWRfMyIsMix7InJhZGl1cyI6LTN9XV0=
\[\begin{tikzcd}
	A && B \\
	\\
	&& C &&&&& {}
	\arrow["{id_1}", from=1-1, to=1-1, loop, in=55, out=125, distance=10mm]
	\arrow["f", from=1-1, to=1-3]
	\arrow["{g \circ f}"', from=1-1, to=3-3]
	\arrow["{id_2}", from=1-3, to=1-3, loop, in=55, out=125, distance=10mm]
	\arrow["g", from=1-3, to=3-3]
	\arrow["{id_3}"', from=3-3, to=3-3, loop, in=305, out=235, distance=10mm]
\end{tikzcd}\]
\column{0.7\textwidth}
\begin{itemize}
 \item
A \textit{category} $\bC$ consists of:
\begin{itemize}
 \pause\item a collection $|\bC|$ of \textit{objects};
 \pause\item for all $A, B \in |\bC|$, a collection $\bC(A, B)$ of \textit{morphisms} from $A$ to $B$;
 \pause\item for all $A \in |\bC|$, an \textit{identity morphism} $id_A \in \bC(A, A)$;
 \pause\item for each pair of morphisms $g \in \bC(A,B)$, $f \in \bC(A,B)$, a morphism $g \circ f \in \bC(A,C)$, such that composition is associative.
\end{itemize}
\pause\item If $\bC(A, B)$ is a set, we call it the \textit{homset} from $A$ to $B$.
\pause\item For every category $\bC$, the \textit{opposite category} $\bC^{op}$ is a category.
\pause\item For every pair of categories $\bC$, $\bD$, the \textit{product category} $\bC \times \bD$ is a category.
\end{itemize}
\end{columns}
\end{frame}
\begin{frame}[fragile]{Functors}
	\begin{columns}
		\column{0.3\textwidth}
		\[\begin{tikzcd}
			A && B \\
			\\
			{F(A)} && {F(A)}
			\arrow["{f\ \in\ \mathbb{C}(A,B)}", from=1-1, to=1-3]
			\arrow["{F}"', dotted, from=1-1, to=3-1]
			\arrow["{F}", dotted, from=1-3, to=3-3]
			\arrow["{F(f)}"', from=3-1, to=3-3]
		\end{tikzcd}\]
		\column{0.7\textwidth}
		A \textit{functor} $F : \bC \to \bD$ is a structure-preserving map between two categories:
		\begin{itemize}
			\pause\item $F$ maps an object $A \in |\bC|$ to an object $B \in |\bD|$
			\pause\item $F$ maps a morphism $f \in \bC(A,B)$ to a morphism $F(f) \in \bD(F(A),F(B))$
			\pause \item $F(id_A) = id_{F(A)}$
			\pause \item $F(g \circ f) = F(g) \circ F(f)$
		\end{itemize}
		Functors from a category into itself are known as \textit{endofunctors}.
\end{columns}
\end{frame}
\begin{frame}[fragile]{Natural Transformations}
	\begin{columns}
		\column{0.3\textwidth}
		\[\begin{tikzcd}
			A && B \\
			{F(A)} && {F(B)} \\
			\\
			{G(A)} && {G(B)}
			\arrow["f", from=1-1, to=1-3]
			\arrow["{F(f)}", from=2-1, to=2-3]
			\arrow["{\phi_A}"{description}, from=2-1, to=4-1]
			\arrow["{\phi_B}"{description}, from=2-3, to=4-3]
			\arrow["{G(f)}", from=4-1, to=4-3]
		\end{tikzcd}\]
		\column{0.7\textwidth}
	    \begin{itemize}
	     \item Structure-preserving maps between functors.
	     \begin{itemize}
	      \pause\item Let $F,G : \bC \to \bD$ be functors.
	      \pause\item A natural transformation $\phi$ is an \textit{indexed family of morphisms} - for every object $A \in |\bC|$,  $\phi_A$ is a morphism from $F(A)$ to $G(A)$.
	      \pause\item These morphisms satisfy the \textit{naturality condition}:
	      \begin{align*}
	       \forall f \in \bC(A,B) : \phi_B \circ F(f) = G(f) \circ \phi_A
	      \end{align*}
	     \end{itemize}
	     \pause\item Given two functors $F$ and $G$, we write the collection of all natural transformation between them as $\text{Nat}(F, G)$.
	    \end{itemize}
	\end{columns}
\end{frame}
\begin{frame}{Exercise 1 : Order Categories}
\begin{itemize}
	\item[a)] Let $\leq$ be a reflexive, transitive order (a \textit{preorder}) on a set $M$. Show that if we define objects by $|\bPre(M, \leq)| = M$ and morphisms by $\exists! f_{x \leq y} \in \bPre(x,y) \Leftrightarrow x \leq y$, then $\bPre(M, \leq)$ forms a category.
	\item[b)] Let $F : \bM \to \bM$ be an endofunctor on $\bM$. Show that $F$ defines a monotonic function $M \to M$, i.e. $\forall x,y : x \leq y \implies F(x) \leq F(y)$.
	\item[c)] Let $F,G : M \to M$ be monotonic functions. Let $\phi$ be a natural transformation $F \to G$. Show that $\forall x \in M: F(x) \leq G(x)$.
\end{itemize}
\end{frame}
\begin{frame}[fragile]{Exercise 1 : Order Categories, Solution a)}
	\begin{columns}
		\column{0.3\textwidth}
		% https://q.uiver.app/#q=WzAsMyxbMCwwLCJ4Il0sWzIsMCwieSJdLFsyLDIsInoiXSxbMCwxLCJmX3t4IFxcbGVxIHl9IiwyXSxbMSwyLCJnX3t5IFxcbGVxIHp9IiwyXSxbMCwyLCIoZyBcXGNpcmMgZilfe3ggXFxsZXEgen0iLDJdLFswLDAsInggXFxsZXEgeCJdLFsxLDEsInkgXFxsZXEgeSJdLFsyLDIsInogXFxsZXEgeiIsMix7InJhZGl1cyI6LTN9XV0=
		\[\begin{tikzcd}
			x && y \\
			\\
			&& z
			\arrow["{x \leq x}", from=1-1, to=1-1, loop, in=55, out=125, distance=10mm]
			\arrow["{{x \leq y}}"', from=1-1, to=1-3]
			\arrow["{{x \leq z}}"', from=1-1, to=3-3]
			\arrow["{y \leq y}", from=1-3, to=1-3, loop, in=55, out=125, distance=10mm]
			\arrow["{{y \leq z}}"', from=1-3, to=3-3]
			\arrow["{z \leq z}"', from=3-3, to=3-3, loop, in=305, out=235, distance=10mm]
		\end{tikzcd}\]
		\column{0.7\textwidth}
		\begin{itemize}
			\item $\leq$ is reflexive, so we have $\forall x : x \leq x \implies \exists id_{x \leq x} \in \bPre(x,x)$.
			\item Because of transitivity, for every pair of morphisms $f_{x \leq y}$ and $g_{y \leq z}$, we have a composed morphism $(g \circ f)_{x \leq z}$.
			\item Since our morphisms are just witnesses of an ordering, they dont care about the order of function application, so composition is associative.
		\end{itemize} 
	\end{columns} 
\end{frame}
\begin{frame}[fragile]{Exercise 1 : Order Categories, Solution b)}
	\begin{columns}
		\column{0.5\textwidth}
		% https://q.uiver.app/#q=WzAsNCxbMCwwLCJ4Il0sWzMsMCwieSJdLFswLDIsIkYoeCkiXSxbMywyLCJGKHkpIl0sWzAsMiwiRiJdLFsxLDMsIkYiXSxbMCwxLCJmIDogeCBcXGxlcSB5IiwxXSxbMiwzLCJGKGYpIDogRih4KSBcXGxlcSBGKHkpIiwxXV0=
		\[\begin{tikzcd}
			x &&& y \\
			\\
			{F(x)} &&& {F(y)}
			\arrow["{{x \leq y}}"{description}, from=1-1, to=1-4]
			\arrow["F", from=1-1, to=3-1]
			\arrow["F", from=1-4, to=3-4]
			\arrow["{{F(x) \leq F(y)}}"{description}, from=3-1, to=3-4]
		\end{tikzcd}\]
		\column{0.5\textwidth}
		\begin{itemize}
			\item By the definition of functors, $F$ must take each morphism $f_{x \leq y} \in \bPre(x,y)$ to a morphism $F(f)_{F(x) \leq F(y)} \in \bPre(F(x),F(y))$.
		\end{itemize} 
	\end{columns} 
\end{frame}
\begin{frame}[fragile]{Exercise 1 : Order Categories, Solution c)}
	\begin{columns}
		\column{0.3\textwidth}
		% https://q.uiver.app/#q=WzAsNCxbMCwwLCJ4Il0sWzMsMCwieSJdLFswLDIsIkYoeCkiXSxbMywyLCJGKHkpIl0sWzAsMiwiRiJdLFsxLDMsIkYiXSxbMCwxLCJmIDogeCBcXGxlcSB5IiwxXSxbMiwzLCJGKGYpIDogRih4KSBcXGxlcSBGKHkpIiwxXV0=
		\[\begin{tikzcd}
			F(x)\\
			\\
			{G(x)}
			\arrow["\phi_x\ :\ F(x) \leq G(x)", from=1-1, to=3-1]
		\end{tikzcd}\]
		\column{0.7\textwidth}
		\begin{itemize}
			\item By the definition of natural transformations, for every object $x$, $\phi_x$ is a morphism $F(x) \to G(x)$. If such a morphism exists, we have $F(x) \leq G(x)$.
		\end{itemize} 
	\end{columns} 
\end{frame}