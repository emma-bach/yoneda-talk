\begin{frame}{Naturality from Polymorphism}
	\begin{itemize}
		\item The naturality condition resembles an equality we saw a few weeks ago:
		\begin{align*}
			r_B \circ \texttt{map}(a) = \texttt{map}(a) \circ r_A
		\end{align*}
		\pause \item This is the free theorem we got for a parametrically polymorphic function \texttt{r :: [X] -> [X]} and an arbitrary function \texttt{a : A -> B}.
		\pause \item This free theorem is equivalent to the statement that $r$ is a natural transformation.
	\end{itemize}
\end{frame}
\begin{frame}{Naturality from Polymorphism}
    \begin{itemize}
      \item In general, assume we have:
      \begin{itemize}
        \pause\item two functors \texttt{F} and \texttt{G},
        \pause\item a parametrically polymorphic function \texttt{r : F x -> G x},
        \pause\item an arbitrary function \texttt{f : A -> B}.
      \end{itemize}
      \pause\item Then we get the following free theorem:
      \begin{equation*}
       \texttt{r . fmap f = fmap f . r}
      \end{equation*}
      %\item[] (Where the fmap on the left belongs to \texttt{F} and the one on the right belongs to \texttt{G})
      \vspace{-18pt}
      \pause\item In categorical notation:
      \begin{equation*}
        r_B \circ F(f) = G(f) \circ r_A
      \end{equation*}
      \vspace{-18pt}
      \pause\item So our free theorem is a proof that any parametrically polymorphic function \texttt{r} is a natural transformation!
      \pause\item It turns out that parametrically polymorphic functions correspond exactly to natural transformations between endofunctors $\bS et \to \bS et$.
    \end{itemize}
\end{frame}
\begin{frame}[fragile]{Homfunctors}
	\begin{columns}
		\column{0.3\textwidth}
		% https://q.uiver.app/#q=WzAsMyxbMCwwLCJBIl0sWzIsMiwiQyJdLFsyLDAsIkIiXSxbMCwxLCJmIFxcY2lyYyBnIiwxXSxbMCwyLCJmIiwxXSxbMiwxLCJnIiwxXV0=
		\[\begin{tikzcd}
			A && B \\
			\\
			&& C
			\arrow["f"{description}, from=1-1, to=1-3]
			\arrow["{g \circ f}"{description}, from=1-1, to=3-3]
			\arrow["g"{description}, from=1-3, to=3-3]
		\end{tikzcd}\]
		% https://q.uiver.app/#q=WzAsNCxbNCwwXSxbMCwwLCJcXHtpZF9BXFx9Il0sWzIsMCwiXFx7ZlxcfSJdLFsyLDIsIlxce2YgXFxjaXJjIGdcXH0iXSxbMSwyLCJmIFxcY2lyYyIsMV0sWzIsMywiZyBcXGNpcmMiLDFdLFsxLDMsImYgXFxjaXJjIGcgXFxjaXJjIiwxXV0=
		\[\begin{tikzcd}
			{\{id_A\}} && {\{f\}} && {} \\
			\\
			&& {\{g \circ f\}}
			\arrow["{f \circ}"{description}, from=1-1, to=1-3]
			\arrow["{g \circ f \circ}"{description}, from=1-1, to=3-3]
			\arrow["{g \circ}"{description}, from=1-3, to=3-3]
		\end{tikzcd}\]
		\column{0.7\textwidth}
		\begin{itemize}
			\item For any category $\bC$, a homset $\bC(A,B)$ is a set of morphisms.
			\pause\item We define a functor $\bC(A, -) : \bC \to \mathbb{S}et$:
			\begin{itemize}
				\pause\item $\bC(A, -)$ maps an Object $B$ to the Homset $\bC(A,B)$
				\pause\item A morphism $f : \bC(B,C)$ is mapped to the morphism $f \circ : \bC(A,B) \to \bC(A,C)$
			\end{itemize}
			\pause\item Similarly, $\bC(-, B)$ is a functor $\bC^{op} \to \mathbb{S}et$:
			\begin{itemize}
				\pause\item $\bC(-, B)$ maps an Object $A$ to the Homset $\bC(A,B)$
				\pause\item A morphism $f : \bC(B,A)$ is mapped to the morphism $\circ f: \bC(A,C) \to \bC(B,C)$
			\end{itemize}
		\end{itemize}
	\end{columns}
\end{frame}

\begin{frame}{Functor Categories}
 \begin{itemize}
  \item For any $\bC$, $\bD$, the collection of functors $\bC \to \bD$ form a category.
  \pause\item This category is known as a \textit{functor category} and denoted $\bD^\bC$.
  \pause\item A morphism $\phi \in \bD^\bC(F, G)$ is a natural transformation $F \to G$.
 \end{itemize}
\end{frame}
