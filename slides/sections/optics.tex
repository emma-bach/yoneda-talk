\lstset{
	frame=none,
	xleftmargin=2pt,
	stepnumber=1,
	%numbers=left,
	%numbersep=5pt,
	%numberstyle=\ttfamily\tiny\color[gray]{0.3},
	belowcaptionskip=\bigskipamount,
	captionpos=b,
	escapeinside={*'}{'*},
	language=haskell,
	tabsize=2,
	emphstyle={\bf},
	commentstyle=\it,
	stringstyle=\mdseries\rmfamily,
	showspaces=false,
	keywordstyle=\bfseries\rmfamily,
	columns=flexible,
	basicstyle=\small\sffamily,
	showstringspaces=false,
	morecomment=[l]\%,
}
\begin{frame}[fragile]{Profunctor Optics}
	\begin{itemize}
		\item In functional programming, an \textit{optic} is generally a data structure including some "outer type" \texttt{S} and some "inner type" \texttt{A}.
		\pause\item In general, this involves some sort of \texttt{get} function, which allows access to the inner value, and a \texttt{set} function, which changes the inner value.
		\pause\item \textit{Profunctor optics} are neat and flexible representations of optics as individual polymorphic function.
		\pause\item In particular, profunctor optics make composition of optics trivial.
		\pause\item Equivalence between optics and their profunctor representations comes down to the Yoneda lemma.
	\end{itemize}
\end{frame}
\begin{frame}[fragile]{Adapters}
	\begin{lstlisting}
data Adapter a b s t = Adapter { from :: s -> a, to :: b -> t }
	\end{lstlisting}
	\begin{itemize}
		\pause\item Categorically, this translates to a pair of morphisms $\bSet(S,A) \times \bSet(B,T)$. 
		\pause\item This pair is a morphism $(\bSet^{op} \times \bSet)((A,B), (S,T))$.
		\pause\item We can compose adapters with matching types and define an identity adapter \texttt{Adapter id id}.
		\pause\item This lets us view the category $\bSet^{op} \times \bSet$ as the category $\bAda$ where morphisms are adapters.
	\end{itemize}
\end{frame}
\begin{frame}[fragile]{Profunctors}
	\begin{itemize}
		\item A \textit{profunctor} is a functor $\bC^{op} \times \bC \to \bSet$.
		\pause\item As Haskell code:
		\begin{lstlisting}
class Profunctor p where
	dimap :: (s -> a) -> (b -> t) -> p a b -> p s t
		\end{lstlisting}
		\pause\item The canonical example of a profunctor is the function type former, 
		
		where \texttt{dimap} is function composition:
		\begin{lstlisting}
instance Profunctor (->) where
	dimap f g h = g . h . f
		\end{lstlisting}
		\pause\item We define the category $\bProf$ of Profunctors to be $\bSet^{\bSet^{op} \times \bSet} = \bSet^\bAda$
	\end{itemize}
\end{frame}
\begin{frame}[fragile]{Profunctor Adapters}
	\begin{columns}
		\column{0.3\textwidth}
			% https://q.uiver.app/#q=WzAsNCxbMCwwLCJwKEEsQikiXSxbMCwyLCJwKFMsVCkiXSxbMSwwLCJwJyhBLEIpIl0sWzEsMiwicCcoUyxUKSJdLFswLDEsIkFkYXB0ZXJQX3AiLDJdLFsyLDMsIkFkYXB0ZXJQX3twJ30iLDJdLFswLDIsIlxcZXRhIiwxXSxbMSwzLCJcXHZhcnRoZXRhIiwxXV0=
			\[\begin{tikzcd}
				{p(A,B)} & {p'(A,B)} \\
				\\
				{p(S,T)} & {p'(S,T)}
				\arrow["\eta"{description}, from=1-1, to=1-2]
				\arrow["{AdapterP_p}"', from=1-1, to=3-1]
				\arrow["{AdapterP_{p'}}"', from=1-2, to=3-2]
				\arrow["\vartheta"{description}, from=3-1, to=3-2]
			\end{tikzcd}\]
		\column{0.7\textwidth}
		\begin{itemize}
			\item The profunctor representation of an adapter is given by:
			\begin{lstlisting}
type AdapterP a b s t =
	forall p. Profunctor p -> p a b -> p s t
			\end{lstlisting}
			\pause\item As we saw earlier, the polymorphism makes \texttt{AdapterP} a natural transformation.
			\pause\item We define $\bAdaP$ as the functor category where objects are $|\bAdaP| = |\bAda| = |\bSet^{op} \times \bSet|$, and whose morphisms are profunctor adapters.
			\pause\item Specifically, the homsets in $\bAdaP$ are:
			\begin{align*}
				\bAdaP ((A,B),(S,T)) &= \bSet^\bProf(-(A, B), -(S, T))\\
									 &= \bSet^{\bSet^{\bAda}}(-(A, B), -(S, T))\\
			\end{align*}
		\end{itemize}
	\end{columns}
\end{frame}
\iffalse
\begin{frame}[fragile]{Equivalence of the two representations}
	\begin{itemize}
		\item The Yoneda lemma tells us that for any functor $F$:
		\begin{align*}
			F(A) \simeq \bSet^\bC(\bC(A, =), F)
		\end{align*}
		\pause\item If we remove the reference to any specific functor, we get:
		\begin{align*}
			-(A) \simeq \bSet^\bC(\bC(A, =), -)
		\end{align*}
		\vspace{-18pt}\pause\item So we have:
		\begin{align*}
			&\bAdaP ((A,B),(S,T))\\
			&=\ \bSet^{\bProf}(-(A, B), -(S, T)) =\ \bSet^{\bSet^\bAda}(-(A, B), -(S, T)) \\ 
			&\simeq\ \bSet^{\bSet^\bAda}(\bSet^\bAda(\bAda((A, B), =), -), \bSet^\bAda(\bAda((S, T), =), -))\\
		\end{align*}
	\end{itemize}
\end{frame}
\begin{frame}[fragile]{Equivalence of the two representations}
	\begin{align*}
		&\bAdaP ((A,B),(S,T))\\
		&\simeq\ \bSet^{\bSet^\bAda}(\bSet^\bAda(\bAda((A, B), =), -), \bSet^\bAda(\bAda((S, T), =), -))
	\end{align*}
	\vspace{-16pt}
	\begin{itemize}
		\item The Yoneda Embedding tells us $\bSet^\bAda$
		\begin{align*}
			&\bSet^{\bSet^\bAda}(\bSet^\bAda(\bAda((A, B), =), -), \bSet^\bAda(\bAda((S, T), =), -))\\ 
			&\simeq\  \bSet^\bAda(\bAda((A,B), =), \bAda((S,T), =))
		\end{align*}
		\item The Yoneda Embedding also tells us 
		\begin{align*}
			&\bSet^\bAda(\bAda((A,B), =), \bAda((S,T), =))\\
			&=\ \bSet^\bAda(\bAda((A,B), =), \bAda((S,T), =))\\
			&\simeq\ \bAda((A,B), (S,T))
		\end{align*}
	\end{itemize}
\end{frame}
\fi
\begin{frame}{Equivalence of the representations}
	\begin{align*}
		\bAdaP ((A,B),(S,T)) = \bSet^{\bSet^\bAda}(-(A, B), -(S, T)) \overset{?}{\simeq} \bAda((A,B), (S,T))
	\end{align*}
	\begin{itemize}
		\pause\item So the proof comes down to showing that a set of natural transformations of natural transformations between the application functors is isomorphic to the underlying set of morphisms.
		\pause\item This involves expanding the functors on the left using the Yoneda lemma and then summarizing everything by applying the Yoneda embedding twice.
		\pause\item This proof is left as an exercise to the reader :) \pause(a short version of the proof is uploaded alongside the slides on ilias).
	\end{itemize}
\end{frame}
\begin{frame}{Summary}
	\begin{itemize}
		\item The Yoneda lemma is a fundamental result in category theory.
		\pause\item The lemma states that the set of natural transformations between a homfunctor $\bC(A, -)$ and an arbitrary functor $F : \bC \to \bSet$ is naturally isomorphic to the set $F(A)$.
		\pause\item The Yoneda lemma enables the use of the Yoneda embedding, which is a natural isomorphism between a category $\bC$ and the category of homfunctors on $\bC$.
		\pause\item Equivalence of adapters and profunctor adapters can be shown by applying the Yoneda embedding twice.
		\pause\item Similar techniques can be used to show the equivalence of any optic and its profunctor representation.
	\end{itemize}
\end{frame}