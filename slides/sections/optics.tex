\lstset{
	frame=none,
	xleftmargin=2pt,
	stepnumber=1,
	%numbers=left,
	%numbersep=5pt,
	%numberstyle=\ttfamily\tiny\color[gray]{0.3},
	belowcaptionskip=\bigskipamount,
	captionpos=b,
	escapeinside={*'}{'*},
	language=haskell,
	tabsize=2,
	emphstyle={\bf},
	commentstyle=\it,
	stringstyle=\mdseries\rmfamily,
	showspaces=false,
	keywordstyle=\bfseries\rmfamily,
	columns=flexible,
	basicstyle=\small\sffamily,
	showstringspaces=false,
	morecomment=[l]\%,
}
\begin{frame}[fragile]{Profunctor Optics}
	\begin{itemize}
		\item In functional programming, an \textit{optic} is generally a data structure including some "outer type" \texttt{S} and some "inner type" \texttt{A}.
		\pause\item In general, this involves some sort of \texttt{get} function, which allows access to the inner value, and a \texttt{set} function, which changes the inner value.
		\pause\item \textit{Profunctor optics} are neat and flexible representations of optics as individual polymorphic function.
		\pause\item In particular, profunctor optics make composition of optics trivial.
		\pause\item Equivalence between optics and their profunctor representations comes down to the Yoneda lemma.
	\end{itemize}
\end{frame}
\begin{frame}[fragile]{Adapters}
	\begin{lstlisting}
data Adapter a b s t = Adapter { from :: s -> a, to :: b -> t }
	\end{lstlisting}
	\begin{itemize}
		\item \texttt{s} and \texttt{t} are assumed to be composite types containing values of types \texttt{a} and \texttt{b} respectively.
		\pause\item Categorically, this translates to a pair of morphisms $\bHask(S,A) \times \bHask(B,T)$. 
		\pause\item This pair is a morphism $(\bHask^{op} \times \bHask)((A,B), (S,T))$.
		\pause\item We can compose adapters with matching types and define an identity adapter \texttt{Adapter id id}.
		\pause\item This lets us view the category $\bHask^{op} \times \bHask$ as the category $\bAda$ of adapters.
	\end{itemize}
\end{frame}
\begin{frame}[fragile]{Profunctors}
	\begin{itemize}
		\item A \textit{profunctor} is a functor $\bC^{op} \times \bC \to \bSet$.
		\pause\item As Haskell code:
		\begin{lstlisting}
class Profunctor p where
	dimap :: (s -> a) -> (b -> t) -> p a b -> p s t
		\end{lstlisting}
		\pause\item The canonical example of a profunctor is the function type former, 
		
		where \texttt{dimap} is function composition:
		\begin{lstlisting}
instance Profunctor (->) where
	dimap f g h = g . h . f
		\end{lstlisting}
		\pause\item We define the category $\bProf$ of Profunctors on Haskell types to be the functor category $\bSet^{(\bHask^{op} \times \bHask)} = \bSet^\bAda$
	\end{itemize}
\end{frame}
\begin{frame}[fragile]{Functor Application as a Functor}
	\begin{columns}
		\column{0.3\textwidth}
		\[\begin{tikzcd}
			F && G \\
			\\
			{F(A)} && {G(A)}
			\arrow["{\eta}", from=1-1, to=1-3]
			\arrow["{-(A)}"', dotted, from=1-1, to=3-1]
			\arrow["{-(A)}", dotted, from=1-3, to=3-3]
			\arrow["{\eta_A}"', from=3-1, to=3-3]
		\end{tikzcd}\]
		\column{0.7\textwidth}
		\begin{itemize}
			\item Given a category $\bC$, the operation of applying a functor $F : \bC \to \bSet$ to an object $A \in |\bC|$ is itself a functor from $\bSet^\bC$ to $\bC$.
			\pause\item We write $-(A)$ for this functor.
		\end{itemize}
	\end{columns}
\end{frame}
\begin{frame}[fragile]{Profunctor Adapters}
	\begin{columns}
		\column{0.3\textwidth}
			% https://q.uiver.app/#q=WzAsNCxbMCwwLCJwKEEsQikiXSxbMCwyLCJwKFMsVCkiXSxbMSwwLCJwJyhBLEIpIl0sWzEsMiwicCcoUyxUKSJdLFswLDEsIkFkYXB0ZXJQX3AiLDJdLFsyLDMsIkFkYXB0ZXJQX3twJ30iLDJdLFswLDIsIlxcZXRhIiwxXSxbMSwzLCJcXHZhcnRoZXRhIiwxXV0=
			\[\begin{tikzcd}
				{p(A,B)} & {p'(A,B)} \\
				\\
				{p(S,T)} & {p'(S,T)}
				\arrow["\eta"{description}, from=1-1, to=1-2]
				\arrow["{AdapterP_p}"', from=1-1, to=3-1]
				\arrow["{AdapterP_{p'}}"', from=1-2, to=3-2]
				\arrow["\vartheta"{description}, from=3-1, to=3-2]
			\end{tikzcd}\]
		\column{0.7\textwidth}
		\begin{itemize}
			\item The profunctor representation of an adapter is given by:
			\begin{lstlisting}
type AdapterP a b s t =
	forall p. Profunctor p -> p a b -> p s t
			\end{lstlisting}
			\pause\item As we saw earlier, the polymorphism makes \texttt{AdapterP} a natural transformation.
			\pause\item We define $\bAdaP$ as the functor category whose objects are those of $\bAda = \bHask^{op} \times \bHask$, but whose morphisms are profunctor adapters.
			\pause\item The homsets in $\bAdaP$ are defined as:
			\begin{align*}
				\bAdaP ((A,B),(S,T)) &= \bSet^\bProf(-(A, B), -(S, T))\\
			\end{align*}
		\end{itemize}
	\end{columns}
\end{frame}
\begin{frame}[fragile]{Equivalence of the two representations}
	\begin{itemize}
		\item The Yoneda lemma tells us that for any functor $F$:
		\begin{align*}
			F(A) \simeq \bSet^\bC(\bC(A, -), F)
		\end{align*}
		\pause\item If we remove the reference to any specific functor, we get:
		\begin{align*}
			-(A) \simeq \bSet^\bC(\bC(A, =), -)
		\end{align*}
		\vspace{-18pt}\pause\item So we have:
		\begin{align*}
			&\bAdaP ((A,B),(S,T))\\
			&=\ \bSet^{\bProf}(-(A, B), -(S, T))\\ 
			&\simeq\ \bSet^{\bProf}(\bProf(\bAda((A, B), =), -), \bProf(\bAda((S, T), =), -))\\
		\end{align*}
	\end{itemize}
\end{frame}
\begin{frame}[fragile]{Equivalence of the two representations}
	\begin{align*}
		&\bAdaP ((A,B),(S,T))\\
		&\simeq\ \bSet^{\bProf}(\bProf(\bAda((A, B), =), -), \bProf(\bAda((S, T), =), -))
	\end{align*}
	\vspace{-16pt}
	\begin{itemize}
		\item Applying the Yoneda embedding of the functor category $\bProf = \bSet^{\bAda}$ gives:
		\begin{align*}
			&\bSet^{\bProf}(\bProf(\bAda((A, B), =), -), \bProf(\bAda((S, T), =), -))\\ 
			&\simeq\  \bProf(\bAda((A,B), =), \bAda((S,T), =))
		\end{align*}
		\item Applying the Yoneda embedding again, this time of the category $\bAda$, gives:
		\begin{align*}
			&\bProf(\bAda((A,B), =), \bAda((S,T), =))\\
			&=\ \bSet^\bAda(\bAda((A,B), =), \bAda((S,T), =))\\
			&\simeq\ \bAda((A,B), (S,T))
		\end{align*}
	\end{itemize}
\end{frame}
\begin{frame}{Summary}
	\begin{itemize}
		\item The Yoneda lemma is a fundamental result in category theory.
		\pause\item The lemma states that the set of natural transformations between a homfunctor $\bC(A, -)$ and an arbitrary functor $F : \bC \to \bSet$ is naturally isomorphic to the set $F(A)$.
		\pause\item The Yoneda lemma enables the use of the Yoneda embedding, which is a natural isomorphism between a category $\bC$ and the category of homfunctors on $\bC$.
		\pause\item Using the Yoneda embedding twice, we have shown the equivalence of adapters and profunctor adapters.
		\pause\item Similar techniques can be used to show the equivalence of any optic and its profunctor representation.
	\end{itemize}
\end{frame}