\begin{frame}{Cayley's Theorem}
\begin{itemize}
	\item Every group $(G, *, e)$ is isomorphic to a subgroup of the group of permutations of $G$.
	\pause\item Specifically:
	\begin{itemize}
		\item One side of the bijection is constructed by sending $g \in G$ to the permutation which maps $f_g : x \mapsto g * x$
		\item The other side sends a permutation $f$ to the element $f(e)$
	\end{itemize}
	\pause\item The Yoneda lemma is often viewed as a generalization of Cayley's theorem.
\end{itemize}
\end{frame}
\begin{frame}{Exercise 2 - Cayley's Theorem for Monoids}
	Use the Yoneda embedding to show that every monoid $M$ is isomorphic to a monoid of functions $M \to M$. 
	\\
	\textbf{Hint 1:} The Yoneda embedding gives an isomorphism between objects and their homfunctors. \\
	\textbf{Hint 2:} Two weeks ago we saw that every monoid $M$ defines a category $\bM$ with a single object $*$ and a morphism $m$ for each element $m \in M$, where we define morphism composition to be the monoid operation.
\end{frame}
\begin{frame}[fragile]{Exercise 2 - Cayley's Theorem for Monoids}
	\begin{columns}
		\column{0.3\textwidth}
		\[\begin{tikzcd}
			* && * \\
			\\
			{\bM(*, -)} && {\bM(*,-)}
			\arrow["{m\ \in\ \bM(*,*)}", from=1-1, to=1-3]
			\arrow["{\mathcal{Y}}"', dotted, leftrightarrow, from=1-1, to=3-1]
			\arrow["{\mathcal{Y}}", dotted, leftrightarrow, from=1-3, to=3-3]
			\arrow["{\mathcal{Y}(m) = m \circ}"', from=3-1, to=3-3]
		\end{tikzcd}\]
		\column{0.7\textwidth}
		\begin{itemize}
			\item The Yoneda embedding is an isomorphism mapping each object to its homfunctor.
			\pause\item We only have one object $*$, and thus only one homfunctor $\bM(*, -)$.
			\pause\item Each element $m \in M$ is a morphism. By the definition of the homfunctor, this morphism is mapped to the set function 
			\begin{align*}
				\bM(*, m) = m \circ : M &\to M\\
				n &\mapsto m \circ n
			\end{align*}
			\vspace{-20pt}\pause\item Thus, the Yoneda embedding on $M$ is an isomorphism between monoid objects and a set of functions $M \to M$. These functions form a monoid under composition.\\\hfill $\qed$ 
		\end{itemize}
	\end{columns}
\end{frame}
