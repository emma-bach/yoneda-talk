\begin{frame}{Cayley's Theorem}
\begin{itemize}
	\item Every group $(G, *, e)$ is isomorphic to a subgroup of the group of permutations of $G$.
	\pause\item Specifically:
	\begin{itemize}
		\item One side of the bijection is constructed by sending $g \in G$ to the permutation which maps $f_g : x \mapsto g * x$
		\item The other side sends a permutation $f$ to the element $f(e)$
	\end{itemize}
	\pause\item The Yoneda lemma is often viewed as a generalization of Cayley's theorem.
\end{itemize}
\end{frame}
\begin{frame}{Exercise 1 - Cayley's Theorem for Monoids}
	Use the Yoneda embedding to show that every monoid $M$ is isomorphic to a monoid of functions $M \to M$. 
	\\
	\textbf{Hint 1:} The Yoneda embedding gives an isomorphism between objects and their homfunctors. \\
	\textbf{Hint 2:} Two weeks ago we saw that every monoid $M$ defines a category with a single object $*$ and a morphism $m$ for each element $m \in M$. 
\end{frame}
\begin{frame}[fragile]{Exercise 1 - Cayley's Theorem for Monoids}
	\begin{columns}
		\column{0.3\textwidth}
		\[\begin{tikzcd}
			A && B \\
			\\
			{\mathbb{C}(A, -)} && {C(B,-)}
			\arrow["{f\ \in\ \mathbb{C}(A,B)}", from=1-1, to=1-3]
			\arrow["{\mathcal{Y}}"', dotted, leftrightarrow, from=1-1, to=3-1]
			\arrow["{\mathcal{Y}}", dotted, leftrightarrow, from=1-3, to=3-3]
			\arrow["{\mathcal{Y}(f)}"', from=3-1, to=3-3]
		\end{tikzcd}\]
		\column{0.7\textwidth}
		\begin{itemize}
			\item The Yoneda embedding is an isomorphism mapping each object to its homfunctor.
		\end{itemize}
	\end{columns}
\end{frame}
